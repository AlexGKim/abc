\documentclass[preprint]{aastex}
\usepackage{amsmath,amssymb}
\usepackage{mathrsfs}
\usepackage{graphicx}
\usepackage{natbib}
\bibliographystyle{apj}

\newcommand{\mli}[1]{\mathit{#1}}
%\usepackage{epstopdf}

\begin{document}

\title{Type Ia Supernova Model}
\author{Alex Kim}

\section{Model}
We infer a Hubble diagram based on measurements of discovered transients
classified as Type~Ia supernovae (SNe~Ia).  The inferred Hubble diagram can be
expressed as
\begin{equation}
P(\vec{\mu},\vec{z} |  \vec{\hat{D}}=1, \vec{\hat{T}}=\text{SN~Ia}, \vec{\hat{ADU}}, \vec{\hat{T}}_S,\vec{\hat{z}}_S,
\vec{\hat{\theta}}_S,\vec{\hat{z}}_H,\vec{\hat{\theta}}_H),
\label{hd:eqn}
\end{equation}
where the inferred quantities are the set of distance moduli ($\vec{\mu}$)
and redshifts ($\vec{z}$), and 
the measurements include: $\hat{D}$ the boolean that indicates that
the transient was discovered; $\hat{ADU}$ the transient photometry;
spectroscopically derived transient
typing, redshift and parameters $\hat{T}_S$, $\hat{z}_S$, $\hat{\theta}_S$;
and the redshift and parameters of the associated host galaxy $\hat{z}_{H}$, $\hat{\theta}_H$.
A classifier takes these data and infers types $\vec{\hat{T}}$.
The conditions $\vec{\hat{D}}=1$, $\vec{\hat{T}}=\text{SN~Ia}$ hold as only transients
discovered and classified as SNe~Ia are modeled.  We use the convention of using carets $\hat{}$
to  label measurements.

An alternative would have been to
model measurements of all discovered transients, not just those typed as SN~Ia. 
The non-Ia transient population has significant model uncertainty and the information they impart on the expansion history is weak:  I anticipate it is cleaner to limit ourselves to
modeling discoveries we think are SN~Ia. This restriction does not obviate the need
to consider non-Ia contamination in the sample, rather
we only need to model non-Ia's
that are mistaken for Ia and not all discovered transients. 

Here $\mu$ is modeled; most analyses instead
directly have cosmological parameters $\{\Omega_M, \Omega_\Lambda, w_0, w_a\}$.
A motivation is to provide results that can be applied to test arbitrary cosmological models:
indeed the model-dependence of the $\mu$'s reported for high-$z$ supernovae 
\citep[e.g.][]{2012ApJ...746...85S} is a source of confusion to the community.
%An example of a non-parametric model for $\mu$ that is independent of cosmological
%parameters is given in the Appendix of
%\citet{2013ApJ...764..116W}.

DES plans a ``photometry''-only analysis, in that a significant subset of transients used
in the Hubble diagram will not observed spectroscopically while active.  Therefore,
our model must account for transient mistyping and host-galaxy mismatches.  To obtain
our targeted
accuracy we need to account for covariances that occur due to photometric
calibration and Malmquist bias.  Our model will contain elements that may not
have been included in previous analyses. 

A sketch of a model explaining the data is depicted through the Probabilistic Graphical Model
(PGM)
shown in Figure~\ref{pgm:fig}.

\begin{figure}[htbp] %  figure placement: here, top, bottom, or page
   \centering
   \includegraphics[width=7in]{/Users/akim/project/abc/results/hdpgm.eps} 
   \caption{Probabilistic Graphical Model for the SN~Ia analysis.  The forward modeling
   flow goes from left to right.  Starting from the position and distance
   $\mu$, $z$, $\text{RA/Dec}$ and galaxies $G$, the transient type $T$ and host-galaxy
   $\theta_{gi}$ are realized. 
   The transient
    parameters $\theta_T$, $\theta_{Ti}$ along
   with redshift and distance modulus determine the incoming photon flux $n$, $n_g$.
   The instrumental transmission function $\phi$ gives the expected and realized
   counts $\mathit{ADU}$, $\mathit{ADU}_g$, $\mathit{ADU}_V$.  The calibration measurements
   are $Z$.  The measured galaxy catalog is $\hat{\mathit{Gals.}}$  Only detected
   supernovae $\hat{D}$ have subsequent light curve ($\hat{ADU}$), spectral
   $\hat{X}_S$, and associated host galaxy $\hat{X}_H$ measurements.  A supernova
   classifier takes measurements and infers a type $\hat{T}$, which is used for sample selection.
   \label{pgm:fig}}
\end{figure}


\subsection{Forward Model}
Let us start with the model description of the universe and its contents, specifically transients
that are discovered as typed as SN~Ia.
\begin{itemize}
\item $G$ is the set of all galaxies that can host a transient.  It is
the union of the observed galaxy catalog plus potential galaxies that have not
met the detection threshold, presumably due to low surface brightness.
\item $\text{RA}=\hat{\text{RA}}$, $\text{Dec} =\hat{\text{Dec}}$.  The observed
coordinates are taken to be the true coordinates of the transient.
\item $P(g_i | G, \text{RA/Dec})$. 
The association of a host galaxy labeled by $g_i$ with the transient.
This pdf is
the deliverable we want from Ravi.
\item $P(z_i, \theta_{gi} | g_i)$. 
The redshift and galactic parameters associated with SN results.
For observed galaxies the parameter distributions 
are based on measurements e.g.\ $\theta_{gi} \sim \mathcal{N}(\hat{\theta}_{gi},\sigma^2_{\hat{\theta}_{gi}})$, otherwise the distribution
of $\theta_{gi}$ for ``hostless'' galaxies can be fit as part of the model.
\item $\mu_i=\mu(z_i; \theta_\mu)$.  The distance modulus is a function of redshift,
we don't impose a specific form for  $\mu$ at the moment,  it could
be a $\Lambda$CDM prediction with cosmological parameters, a spline with knot values
as parameters, or as extreme as independent values for each $z$.
The redshift of the galaxy is assumed to be that of the transient.
\item $P(T_i | \theta_{gi})$.  The transient type, SN~Ia or non-Ia, depends
on the host properties.  We could make the relative rates be dependent
on specific star formation rate and morphology.
\item $P(\theta_T^{Ia}, \theta_{Ti}^{Ia})$.  The model parameters for SNe~Ia, including
parameters that describe distributions.
Different SN~Ia models could be considered: the treatment of SALT2 parameters
has been presented in \citet{2011MNRAS.418.2308M}.
The SALT2 SN~Ia
model has individual $c$ and $x_1$ parameters that would constitute $\theta_{Ti}^{Ia}$,
and global $\alpha$, $\beta$, and intrinsic dispersion  parameters  common to all supernovae
 and  underlying  $c$ and $x_1$ distributions that would constitute $\theta_T^{Ia}$.
\item $P(\theta_T^{non-Ia}, \theta_{Ti}^{non-Ia})$.  Non-Ia transients
that can be mistyped as Ia are described by a parameterized model.
\item $L_i(t,\lambda; T_i, \theta_T^{Ia}, \theta_{Ti}^{Ia}, \theta_T^{non-Ia}, \theta_{Ti}^{non-Ia},
\theta_{gi})$.  The source model specifies
the luminosity. The  model includes  information on the
source and line-of-sight effects, which specify the effective SED.   Only the
parameters of the model of the appropriate type $T$ are relevant.  The galaxy parameters
$\theta_{gi}$ are included if correlations with host-galaxy properties are to be modeled.
\item $n_i(t,\lambda; L_i, \mu_i, z_i)$.  The  flux incident at Earth
is specified by the luminosity, distance modulus, and redshift.
\end{itemize}

The universe is observed with an experiment.
\begin{itemize}
\item The transmission functions $\phi(\lambda)$ describe the optical path from the
atmosphere to counts, and is calibrated with measurements $\hat{Z}$.  We expect
that calibration group to provide us with this function and its uncertainties. 
\item $\mathit{ADU}$ and
$\mathit{ADU}_g$ are the expected counts of the transient and galaxy respectively,
coming from the combination of source flux $n$ and transmission $\phi$.
\item $\mathit{ADU}_V$ is the realized flux.
\item The transient is discovered $\hat{D}$ depending on the realized $\mathit{ADU}_V$.
This should depend mostly on signal-to-noise.
\item $\hat{\mathit{ADU}}$ and
$\hat{\mathit{ADU}}_g$ are measured light curves.
\item $\hat{T}_S$, $\hat{z}_S$, $\hat{\theta}_S$ Transient redshift and properties from
spectroscopic data. I skip the details of the actual measurement  and make a direct association
with the fluxes of the transient and underlying galaxy $n$, $n_g$.

\end{itemize}
has, which determine the expected counts of the transient and galaxy,    These expected counts in turn determine realized  transient counts 
The signal to noise of the realized counts specify the probability of discovery
$\hat{D}$ .    The photometry of the subset of discovered transients
are denoted as $\hat{\mathit{ADU}}$.

For discovered transients, there may be spectra of the object itself
from which typing, redshift and spectral parameters $\hat{T}_S$, $\hat{z}_S$, $\hat{\theta}_S$ are derived.
From amongst the detected galaxies $\hat{\mathit{Gals.}}$  a host is associated with the transient
based on its RA/Dec; the host information is summarized as the
host redshift (spectroscopic or photometric) and parameters $\hat{z}_{H}$, $\hat{\theta}_H$.

The observatory transmission $\phi$ are calibrated $\hat{Z}$.


\subsection{Backward Model}
 
I now systematically
examine the model going backwards, from the observables to the fundamental parameters.
This is useful for understanding dependencies in constructing the likelihood.

The observables are:
\begin{description}
\item[$\hat{\mathit{Gals}}$] Detected galaxies from which a transient's host is identified.
Depends on the model for galaxies $\theta_G$.
\item[$\hat{D}$] Transient detection.  It depends on
the realized signal $\hat{\mathit{ADU}_V}$ and its noise, mostly through the signal-to-noise
$\mathit{SNR}$.
\item[$\hat{\mathit{ADU}}$]  Photometry of discovered transients.  Its inclusion
depends on discovery $\hat{D}$ and has the value of the realized signal $\mathit{ADU}_V$.
\item[$\hat{z}_H$, $\hat{\theta}_H$] Host galaxy redshift and properties from spectroscopic
and photometric data, based on association
of the host from the pool of detected galaxies $\hat{\mathit{Gals}}$ and the transient RA/Dec.
\item[$\hat{T}_S$, $\hat{z}_S$, $\hat{\theta}_S$] Transient redshift and properties from
spectroscopic data. I skip the details of the actual measurement  and make a direct association
with the fluxes of the transient and underlying galaxy $n$, $n_g$.
\item[$\hat{Z}$] Calibration zeropoints.  Depends on the calibration process
and the underlying transmission functions $\phi(\lambda)$.  Specifies
the transmission functions.
\end{description}

The following are elements that are to be modeled, including ``nuisance'' parameters.
\begin{description}
\item[$\mathit{ADU}_V$]  Realized counts
of the transient.  Depends
on the expected counts of the transient and host $\mathit{ADU}$, $\mathit{ADU}_g$.
Although all transients will have realized counts but not all will be measured, hence the distinction
from $\hat{\mathit{ADU}}$.
\item[$\mathit{ADU}$, $\mathit{ADU}_{g}$]  Expected counts
of the transient and background galaxy respectively.  Depends
on the incident fluxes $n$, $n_g$ and the instrumental
transmission $\phi(\lambda)$.
\item[$\phi(\lambda)$] DES transmission functions.
\item[$n$, $n_g$] The incident fluxes at Earth of the transient and background
galaxy.  Depends on  $z$ and $\mu$, and the SEDs of the Type and Host as
specified by $\theta_{Ti}$, $\theta_{Gi}$, $\theta_{T}$, $\theta_{G}$.
\item[$\theta_{Ti}$, $\theta_{Gi}$] Parameters that describe the properties of the individual transient
and host respectively.
\item[$\theta_{T}$, $\theta_{G}$] Parameters that describe the global properties of the transient
and host models.
\item[$T$, $G$]  The type of transient and projected galaxies (referred to as host), which
specifies which model to use. The realization depends on the redshift $z$.
\item[$\vec{\mu}$, $\vec{z}$]  The true distance moduli and cosmological redshifts under
consideration.  Each redshift has a unique distance modulus.
\item[$\mathbf{RA/Dec}$] The coordinate of the transient.  For us, it is sufficient to directly
assign the coordinate measured
from the images.
\end{description}

\section{Subsections of the Model}

\subsection{Type}
\label{type:sec}
Since our transient sample may not consist purely of SNe~Ia, the model
has a term of the form $P(T, G | z,\mu)$, which should account for all types of
objects that could potentially be mistaken for Type~Ia:
the rates and association with host galaxies are described for
the full population of potential interlopers.
This term is not currently well known, particularly at the highest redshifts probed by DES.

An idea is to consider two types: SN~Ia and non-SN~Ia, where the latter's
intrinsic distributions have loose priors.  Transients that have been spectroscopically
classified as non-Ia will provide the strongest leverage in constraining the
contamination: this subset must be unbiased or properly weighted
relative to the underlying population that constitutes the Hubble diagram.
Running the model on a pure sample of SNe~Ia can have benefits.
As discussed in \S\ref{systematics:sec}, comparing these results with
those from the full sample provides
a test of systematics.  Alternatively, the non-Ia model may be better constrained
with the pure sample.


To minimize the contribution of non-Ia contamination we strive for
a pure sample selection so that $P(T=\text{non-Ia}| \mu) \rightarrow 0$.


\subsection{Host Matching}
\sloppy
The host-galaxy properties depend on the galaxy catalog and the transient
coordinates $P(\hat{z}_H, \hat{\theta}_H | \hat{\mathit{Gals.}}, \text{RA/Dec})$.  With a
spectrum of the transient and the underlying background, identification of the host galaxy
is usually straightforward.
Otherwise, projections or ambiguous cases can result in the misidentification of
the host.
Other supernova surveys (e.g.\ SNLS) 
with spectroscopically confirmed host associations (e.g.\ with
matched transient and galaxy redshifts) that replicate the DES population
can be used to put a prior on this probability.

%An effective way of addressing this term is to
%analyze the supernovae assuming a correct host match, and
%experimentally determine $P(\text{mismatch} )$ and
%$\left. \langle \hat{\mu}-\mu(z)\rangle \right|_{\text{mismatch}}$, $\left. \langle \hat{z}-z \rangle \right|_{\text{mismatch}}$ and cross-terms to correct bias in the Hubble diagram. 


% only model $T=\text{SN~Ia}$,
%and experimentally determine $P(\text{non-Ia})$, 
%$\left. \langle \mu-\mu_{model} \rangle \right|_{non-Ia}$, and its uncertainty for the
%sample of non-Ia's that are discovered and misclassified as SN~Ia.
%This measurement is achieved through spectroscopic
%typing of a subset drawn from the populations used in the Hubble diagram.
%The weighted bias would be added as a correction to the Hubble diagram under the assumption that all objects are SNe~Ia.



\subsection{Type~Ia Supernova Model}
\label{snmodel:sec}
The DES light curve analysis in principle may be used for typing and determining
redshift, in addition to getting supernova properties.   
Generally the relevant term is
\begin{equation}
P(\hat{D}_i=1,\hat{\mathit{ADU}}_{i}, \hat{T}_{Si}, \hat{z}_{Si}, \hat{\theta}_{Si}, \hat{z}_{Hi}, \hat{\theta}_{Hi}
 | \mu_i,z_i,T_i,\theta_T,\theta_{Ti},\phi(\lambda)).
 \end{equation}
Detection and typing depend on the realized light curves, and hence the truncated
distribution of discovered transients typed as SN~Ia.

The model contains a piece $P(\hat{\mathit{ADU}}_{i} \hat{z}_{Si}| \mu_i,z_i,\theta_T, \theta_{Ti}, ,\phi(\lambda))$, which
is the piece that describes underlying SN~Ia properties and light-curve fits.


\subsection{Detection Efficiency}
There term $P(\hat{D}|\mathit{ADU}_V)$.  For all high-redshift supernova searches
I know of, this depends on the signal-to-noise of the data, and very weakly on the
structure of the underlying galaxy.  Our detection is now based on machine learning,
we should confirm that detection dependencies do not depend on our model parameters
(except for RA/Dec).

\section{Systematic Tests}
\label{systematics:sec}
The Hubble diagrams of subsets of data quantify systematic uncertainty.  
Subsets that may be expected to show evidence for systematics include:
spectroscopically typed SNe~Ia; transients in deep (3) versus shallow (1/2) DES fields;
transients with NIR data; splits based on host properties.

A potentially more probative test is to compare the distance moduii inferred for the same
supernovae, but reducing the amount of data considered.  A subset of DES
transients have higher signal-to-noise, spectroscopic confirmation, and NIR coverage than
the average transient, we can take away that extra information in calculating distance modulus
to get $P(\mu | \mathit{data} - \mu | \mathit{data}^-, z | \mathit{data} - z | \mathit{data}^-)$.
It is possible that differences in $\mu$s and $z$s have smaller uncertainties that the
$\mu$ and $z$ alone due to common measurement uncertainties. 

\bibliography{/Users/akim/Documents/alex}

\end{document}

