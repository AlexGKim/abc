\documentclass[preprint,3p]{elsarticle}
\usepackage{aas_macros}
\usepackage{amsmath,amssymb}
\usepackage{mathrsfs}
\usepackage{graphicx}
\usepackage{bm}
\usepackage{hyperref}

\newcommand{\mli}[1]{\mathit{#1}}
%\usepackage{epstopdf}


\begin{document}

\begin{frontmatter}

\title{Standard Candle Cosmology With Incomplete Spectroscopic Classification}
\author{A.~G. Kim\corref{cor1}}
\ead{agkim@lbl.gov}
\address{Physics Division, Lawrence Berkeley National Laboratory, 1 Cyclotron Road, Berkeley CA, USA 94720}

\begin{abstract}
\end{abstract}
\begin{keyword}
\end{keyword}
\end{frontmatter}

\section{Introduction}
A class of astronomical objects who share a common luminosity are called a standard
candle.  The relative fluxes (or magnitudes) for a set of these objects distributed in space
provide a measurement of their relative luminosity distances. These data, combined with
measurements of the standard candle redshifts, constitute a Hubble
diagram which is used to map the expansion history of the Universe.

Type~Ia supernovae (SNe~Ia) are an example of a standard candle (more precisely a standardizable
candle), whose redshift-magnitude relation provides an accurate expansion history
used to measure the Hubble Constant
\citep{2001ApJ...553...47F} and detect the accelerated expansion of the Universe
\citep{1998AJ....116.1009R, 1999ApJ...517..565P}.  SN~Ia measurements continue
to improve 
\citep{2014A&A...568A..22B} and play a critical role in probing the physics
responsible for the acceleration \citep{2013PhR...530...87W}.
Looking at the present and toward the future, SN~Ia cosmology is
a component of the Dark Energy Survey\footnote{\url{http://www.darkenergysurvey.org}} (DES),
and a science driver in the design of the
Large Synoptic Survey Telescope\footnote{\url{http://www.lsst.org}} (LSST)
and the Wide-Field InfrarRed Survey Telescope
\citep{2015arXiv150303757S}.

The information necessary to construct a Hubble diagram consists of, per object,
its classification as a
member of the standard candle class, its redshift, and its flux.  For SNe~Ia, the flux at
peak brightness is typically determined from multi-band light curves obtained from repeated
photometric observations. Traditionally, spectroscopic data is used to classify the transient
as SN~Ia through the lack of hydrogen and the presence of SiII in its spectrum,
and determine its redshift either directly through the transient spectrum or that of its host
galaxy. While the classification and redshift could be determined from the multi-band
light curves and colors of the host galaxy, the accuracy is insufficient for precision
cosmology \citep{2011ApJ...738..162S}.  Consequently spectroscopic follow-up
has played a critical role in cosmological supernova surveys.

While the large focal planes of the Dark Energy Camera (3 square degrees)
and the LSST (9.6 square degrees) open the
possibility for the photometric discovery and observation of thousands to millions
of SNe~Ia out to $z\sim1$, a corresponding multiplex advantage has not been
achieved for the 10-m
class telescopes used for the spectroscopic classification of faint discoveries:
it is unfeasible to obtain a spectrum for all SN~Ia candidates with well-measured
photometry.

Photometric supernova analysis has been proposed to  take advantage of the statistical weight imported by large numbers of supernova light curves, despite the lack of spectroscopy.
An example of such a program would proceed as follows: A rolling wide-field imaging
survey generates light curves for all transients in the survey area.  In near real-time,
a subset 
of transients from those images are selected for triggered spectroscopic observations
to obtain classifications and redshifts.  At the end of the survey, the
analysis sample is the subset of likely SN~Ia
candidates based on photometric data;  the analysis sample and its subset
with triggered spectroscopy are constructed so as to be drawn from the same parent distribution.
Spectroscopic redshifts are obtained for likely host galaxies.  The Hubble diagram of the
analysis sample is analyzed with the understanding that it may contain multiple populations.

The Sloan Digital Sky Survey-II Supernova Survey is an example of a program
that produced significant numbers of transients
with light curves but without spectroscopic typing \citep{2014arXiv1401.3317S}:
photometric SN analysis has been developed applied to these data \citep{2012ApJ...752...79H,
2013ApJ...763...88C}.   Such analyses are anticipated in the projected
supernova programs of 
DES \citep{2012ApJ...753..152B} and LSST \citep{2012arXiv1211.0310L}.

Current models used for supernova cosmology analysis account for unobserved quantities.
SNe~Ia are not standard but are ``standardizable'' candles, implying that the class may be composed of distinct subpopulations each with its own redshift-dependent rates, magnitude distributions, and secondary
observable (e.g.\ light-curve shape, color) distributions.  Bayesian hierarchical modeling is 
used to account for this extra layer of complexity  to determine characteristics of the SN~Ia population simultaneously
with the cosmology \citep{2011MNRAS.418.2308M,
2015arXiv150701602R}.  Consideration of missing classification is addressed
by
BEAMS \citep{2007PhRvD..75j3508K}, which  includes a non-Ia population in sample, whose luminosity distribution
is inferred along with the cosmology:
BEAMS was used in the analysis of SDSS-II SN data \citep{2012ApJ...752...79H}.

This article addresses the affect of missing spectroscopic typing in
a transient standard candle analysis, with the eventual goal of quantifying
requirements for the spectroscopic subsample in a broader photometric analysis.
The lack of spectroscopy leads to ambiguity in both the transient class and
in the host galaxy due to projected galaxies that appear close to the transient or
the relative low-surface brightness of the host.  The subsample with
spectroscopic typing informs the interpretation of objects lacking spectroscopy.
Differences between observed and
intrinsic distributions in a flux-limited sample can have important effects on the Hubble
diagram  (e.g.\ Malmquist bias) so are accounted for explicitly.
Though motivated by SNe~Ia, this article makes the simplifying assumption that they
are standard candles with random dispersion; the motivation of this article is to understand
the general effects of spectroscopic completeness in a manner that is not specific to
the choice of model for the underlying supernova heterogeneity.

\section{Data Model}
\label{model:sec}
The data in constructing a Hubble diagram and the fundamental parameters of the
model used to describe those data in this article are succinctly summarized in the notation
for the likelihood 
\begin{equation}
p(f, {{T}}_S,{{z}}_S, \theta_G|  \Omega_M, w, \theta_r,\alpha_{Ia},\sigma_{Ia}, \alpha_{\mathit{non-Ia}},\sigma_{\mathit{non-Ia}}),
\label{likelihood:eqn}
\end{equation}
and are sketched as the probabilistic graphical model in Figure~\ref{toypgm:fig}.
The measurements are
\begin{itemize}
\item $f$: the flux at peak brightness, taken 
to be bolometric to facilitate the treatment of
standard candles at different redshift;
\item ${{T}}_S$, ${{z}}_S$: the spectroscopic type and redshift available for a subset of transients;
\item $\theta_{G1}$, $\theta_{G2}$: the spectroscopic redshifts of the galaxies identified
as host and neighbor from imaging.
\end{itemize}
The model parameters are
\begin{itemize}
\item $\Omega_M$, $w$: the mass density and dark-energy equation of state parameters of a flat,
constant-$w$ dark-energy cosmology;
\item $\theta_{r1}$, $\theta_{r2}$:  the parameters
that describe the redshift-dependent relative fraction of SN~Ia's and non-Ia's;
\item $\alpha_{Ia}$, $\sigma_{Ia}$, $\alpha_{\mathit{non-Ia}}$, $\sigma_{\mathit{non-Ia}}$:
the luminosity (times $4\pi$) and intrinsic dispersion for SNe~Ia and non-Ia at peak brightness.
\end{itemize}
Not listed explicitly in Eqn.~\ref{likelihood:eqn} but shown in Figure~\ref{toypgm:fig} are latent model parameters for each transient:
\begin{itemize}
\item $z$: the cosmological redshifts of the host and neighboring galaxies;
\item $T$: the class of the objects, SN~Ia or non-Ia.
\end{itemize}


\begin{figure}[htbp] %  figure placement: here, top, bottom, or page
   \centering
   \includegraphics[width=4.5in]{/Users/akim/project/abc/results/toy_pgm.pdf} 
   \caption{Probabilistic Graphical Model for the  SN~Ia standard candle analysis. The
   parameters include the cosmological parameters $\Omega_M$ and $w$,
   the luminosities $\alpha$ and intrinsic dispersions $\sigma$ of SNe~Ia and non-Ia's,
   and the relative fraction parameters $\theta_{r1}$ and $\theta_{r2}$.  Latent parameters
   include the type $T$ and redshifts of the host and neighboring galaxies $z$ and $z_N$
   for each transient.  The  distance modulus $\mu$ and
   luminosity $L$ are fixed by the other parameters.  The observables include the measured
   flux $f$, spectroscopic type and redshift $T_S$ and $z_S$, and
   the inferred host and neighbor redshifts $\theta_{G1}$ and $\theta_{G2}$.
   The host and neighbor galaxies of all transients are inferred based a common galaxy catalog, 
   and so strictly are not independent from supernova to supernova.
   \label{toypgm:fig}}
\end{figure}


The analysis sample consists of transients that are discovered and pass some sample-selection criteria, for example a photometric-classification score.  While the studies in this
article  explicitly account for
the discovery criterion, they do not do so for the subsequent sample selection.
In other words, sample selection is assumed not to alter the shape of any of the pdf's.

The data for each transient
is modeled according to the following subsets:
{\it obs} are transients observed spectroscopically while active, from which $T_S=1$ are those typed
as SN~Ia and $T_S=0$ as non-Ia.  {\it mis} transients are not observed
spectroscopically when active.  The  {\it obs} and {\it mis} samples
are assumed to be drawn from the same underlying distribution, so that 
same SN~Ia and non-Ia parameters apply for both sets.

\subsection{Model for the {\it obs} set data}
\label{obs:sec}
The likelihood of data from the {\it obs} set, which has
spectroscopic observations $T_S$ and redshift
$z_S$, is described as follow.  The likelihood is
\begin{align}
p(f, {{T}}_S,{{z}}_S|  \Omega_M, w, \theta_r,\alpha_{Ia},\sigma_{Ia}, \alpha_{\mathit{non-Ia}},\sigma_{\mathit{non-Ia}})  &=\sum_T \int_z dz\,
p(f, {{T}}_S,{{z}}_S, T, z| \ldots)\\
&= \sum_T \int_z dz\,
p(f, {{T}}_S,{{z}}_S| T, z,\dots) p(T,z | \ldots).
\end{align}

A direct association of spectroscopic redshift and type are made
\begin{align}
p(z_S|z) &= \delta(z_S-z)\\
p(T_S|T) &= \delta(T_S-T),
\label{specz:eqn}
\end{align}
so that
\begin{align}
p(f, {{T}}_S,{{z}}_S|  \ldots) &= 
p(f| T=T_S, z=z_s,\dots) P(T_S| z= z_S , \ldots) p(z_S|\ldots).
\label{obs:eqn}
\end{align}


The first term describes the expected counts. Given the redshift
and transient type, the brightness of the underlying population
is modeled as being drawn from a standard candle
with mean luminosity $4\pi\alpha_X$ and a normal distribution in magnitude
space with intrinsic dispersion  $\sigma_X$:
\begin{equation}
f| T_S, z_S, \Omega_M, w, \theta_r, \alpha_{Ia},\sigma_{Ia}, \alpha_{\mathit{non-Ia}},\sigma_{\mathit{non-Ia}} \sim \mathcal{N}_{\ln}\left(\ln{\left(\frac{\alpha_{T_S}}{d_L^2(z_S;\Omega_M, w)}\right)}, \frac{\ln{10}}{2.5}\sigma_{T_S}\right),
\label{flux:eqn}
\end{equation}
where $\mathcal{N}_{\ln}$ is the log-normal distribution and $d_L$ is the luminosity distance.
Measurement uncertainty is taken to be negligible.
In a flux limited survey, the pdf's for the data are truncated versions of the underlying
pdf's.
\begin{itemize}
\item For SNe~Ia  $T_S=1$:
\begin{equation}
f | T_S=1, z_S, \Omega_M, w, \alpha_{Ia},\sigma_{Ia} \sim
\frac{\mathcal{N}_{\ln}\left(\ln{\left(\frac{\alpha_{Ia}}{d_L^2(z_S;\Omega_M, w)}\right)}, \frac{\ln{10}}{2.5}\sigma_{Ia}\right)}{P_{Ia}(f > f_0)}.
\label{adusnIa:eqn}
\end{equation}
\item For non-Ia's in set $T_S=0$:
\begin{equation}
f | T_S=0, z_S, \Omega_M, w, \alpha_{\mathit{non-Ia}},\sigma_{\mathit{non-Ia}}\sim 
\frac{\mathcal{N}_{\ln}\left(\ln{\left(\frac{\alpha_{\mathit{non-Ia}}}{d_L^2(z_S;\Omega_M, w)}\right)}, \frac{\ln{10}}{2.5}\sigma_{\mathit{non-Ia}}\right)}{P_{non-Ia}(f > f_0)},
\label{adunonIa:eqn}
\end{equation}
\end{itemize}
where the normalization factor is the complementary cumulative distribution function
for the corresponding log-normal distribution.
It is understood implicitly that as all objects must pass the detection threshold,
the case of $f<f_0$ is never confronted in the analysis.

The second term gives the type probability. In our model,
the underlying relative rates between SNe~Ia and non-Ia is linear in redshift,
 parameterized by the
its values at $z=0$ and $1.1 \times z_{max}$, 
\begin{equation}
\hat{\theta}_r=\theta_{r0}+z\left(\frac{\theta_{r1}-\theta_{r0}}{1.1 \times z_{max}}\right).
\label{rate:eqn}
\end{equation}
The relative rate for SN~Ia discovery for the discovered (truncated) distribution is
\begin{equation}
\theta_r=\frac{\hat{\theta}_rP_{Ia}(f > f_0)}{\hat{\theta}_{r}P_{Ia}(f > f_0) + (1-\hat{\theta}_{r})P_{non-Ia}(f > f_0)}.
\end{equation}
The spectroscopic classifications are drawn from
\begin{equation}
T_S | \Omega_M, w, \theta_{r0}, \theta_{r1} , \alpha_{Ia},\sigma_{Ia}, \alpha_{\mathit{non-Ia}},\sigma_{\mathit{non-Ia}} \sim \text{Bernoulli}(\theta_r).
\end{equation}
Although the underlying type distribution depends only on $ \theta_{r0}$ and $ \theta_{r1}$,  
the observed distribution does depend on the other parameters through $\theta_r$'s dependence on
the two normalization factors
$P_{X}(f > f_0)$.

The third term is the redshift distribution of transient discoveries.
The model takes an underlying redshift distribution of transients in the universe
\begin{equation}
p(z) \propto z^2.
\end{equation}
The redshift distribution of discovered transients is then
\begin{equation}
p(z) = \frac{z^2\left(\hat{\theta}_{r}P_{Ia}(f > f_0) + (1-\hat{\theta}_{r})P_{non-Ia}(f > f_0)
\right)}{\int_{z_{min}}^{z_{max}} dz'\, z'^2\left( 
\hat{\theta}_{r}P_{Ia}(f > f_0) + (1-\hat{\theta}_{r})P_{non-Ia}(f > f_0)
\right)}.
\end{equation}


\subsection{Model for the {\it mis} set data}

Transients in the {\it mis}
set have no spectroscopic confirmation, and their data are treated as follows:
With no redshift from the transient itself, 
the redshift measurement is inferred from potential host galaxies.
In the analysis of this article each transient is associated with two galaxies, one 
with redshift $\theta_{G1}$
that has high probability $\pi_{host}$
of being the host and the other
with redshift $\theta_{G2}$ that has probability $(1-\pi_{host})$.
In the model, the
true host redshift is $z$ and that of the interloping galaxy is $z_N$.
\begin{equation}
P(\theta_{G1},\theta_{G2}|z, z_N) =
	\pi_{host}\delta(z-\theta_{G1})\delta(z_N-\theta_{G2}) +
	(1-\pi_{host}) \delta(z-\theta_{G2})\delta(z_N-\theta_{G1}).
\end{equation}
As for the {\it obs} set, the redshifts of the supernova
 and neighbor are drawn from underlying distributions $p(z) \propto z^2$, $p(z_N) \propto z^2_N$.


The underlying distribution is 
\begin{multline}
p(f, \theta_{G1}, \theta_{G2} | \Omega_M, w, \theta_r, \alpha_{Ia},\sigma_{Ia}, \alpha_{\mathit{non-Ia}},\sigma_{\mathit{non-Ia}})  \\
= \sum_{T}\int_{z,z_N} dz\,dz_N \, p(\mathit{ADU, \theta_{G1}, \theta_{G2}}| z , z_N, T, \ldots)P(T| z, z_N,\ldots) p(z, z_N|\ldots)\\
 =\sum_{T} \int_{z,z_N} dz\,dz_N\, p(f| z ,T, \ldots)p(\theta_{G1}, \theta_{G2}| z , z_N) P(T| z, \ldots) p(z, z_N|\ldots)\\
= \sum_{T} \pi_{host} p(f| z=\theta_{G1} ,T, \ldots) P(T| z=\theta_{G1}, \ldots) p(\theta_{G1},\theta_{G2}|\ldots) \\
 +  (1-\pi_{host}) p(f| z=\theta_{G2} ,T, \ldots) P(T| z=\theta_{G2}, \ldots) p(\theta_{G2},
 \theta_{G1}|\ldots)\\
\propto \sum_{T} \pi_{host} p(f| z=\theta_{G1} ,T, \ldots) P(T| z=\theta_{G1}, \ldots) \\
 +  (1-\pi_{host}) p(f| z=\theta_{G2} ,T, \ldots) P(T| z=\theta_{G2}, \ldots),
\label{adumis:eqn}
\end{multline}
where we take advantage of the fact that
$p(\theta_{G1}, \theta_{G2} | \ldots) = p(\theta_{G2}, \theta_{G1} | \ldots)$ has
no parameter dependence.  The probabilities  for the underlying flux and
relative rate distributions
 Eqns.~\ref{flux:eqn} and \ref{rate:eqn} introduced in \S\ref{obs:sec} apply.

The data likelihood is a truncated version of the
underlying distribution.  
For our analysis, we use the distribution in  Eqn.~\ref{adumis:eqn} normalized by
the complementary cumulative distribution function
\begin{equation}
p(f>f_0, \theta_{G1}, \theta_{G2}|\ldots) = \int_{f_0}^{\infty} p(f, \theta_{G1}, \theta_{G2}|\Omega_M, w, \theta_r, \alpha_{Ia},\sigma_{Ia}, \alpha_{\mathit{non-Ia}},\sigma_{\mathit{non-Ia}})df
\end{equation}
for a detection threshold of $f_0$.



%I highlight the treatment of redshift in this model; the cosmological redshift
%$z$ is set to the values of the
%observed redshifts through $\delta$-functions.  This choice was made specifically to aid
%in the convergence of the Monte Carlo: the discrete possibilities for redshift allow
%for their marginalized likelihood to be used in the calculation of the posterior.
%A more general model would replace the $\delta$-functions
%with coming from a continuous PDF's.
%Peculiar velocities, redshift measurement errors (particularly from
%photometric redshifts) justify this generalization.
%A transient
%without a spectrum has uncertain type and redshift: the redshift
%of the associated host galaxy, and redshifts expected based on the
%flux of the SN~Ia and non-Ia standard candles,
%are viable but disjoint possibilities for the posterior of
%the true underlying redshift. My runs of this scenario produce chains
%that do not mix so I resorted to the simplified treatment presented here.

\section{Instantiation and Analysis of Simulated Data}
\subsection{Simulated Data}
Simulated data sets are realized and analyzed with the model of
 \S\ref{model:sec}.  The generative model of the first data set (Set 1) is
 identical to the model used for analysis, with the following parameter choices:
\begin{itemize}
\item Cosmological parameters $\Omega_M=0.28$, $w=-1$.
\item Type~Ia supernovae with intrinsic dispersion $\sigma_{SNIa}=0.1$.
\item Non-Ia supernovae 2 magnitudes fainter than SNe~Ia with intrinsic
dispersion $\sigma_{non-Ia}=1$.
\item $z_{max}=1.4$.
\item $\theta_{r0}=0.95$, $\theta_{r1}=0.2$ for the relative fraction of SNe~Ia
at $z=0$ and $z=1.1 \times z_{max}$ respectively.
\item 2000 transients total, including discovered and non-discovered,
in the range $0.1<z<z_{max}$.
\item Detection threshold at approximately the mean brightness SN~Ia at $z=1$.
\item Limits of neighbor-galaxy redshifts of $0.1/1.1$ and $1.1\times z_{max}$.
\item $\pi_{host}=0.98$ for the probability of correct host-galaxy assignment.
\item Accumulated spectra for 0\%, 20\%, 60\%, and 100\% of discovered transients.
\end{itemize}

The Hubble diagram for a realization drawn from this model is shown in Figure~\ref{hd:fig}
and a histogram of redshifts for discovered SNe~Ia and non-Ia in Figure~\ref{hist:fig}.
SNe~Ia's are shown in blue and non-Ia's in red, those that fall below detection threshold
are transparent.  The solid points
represent those objects spectroscopically typed for the case of 20\% follow-up.
The green X's represent the 2\% among those without spectroscopy that
are assigned an incorrect redshift.

\begin{figure}[htbp] %  figure placement: here, top, bottom, or page
   \centering
   \includegraphics[scale=0.4]{/Users/akim/project/abc/doc/seed2.pdf}
   \includegraphics[scale=0.4]{/Users/akim/project/abc/doc/seed2_pop2.pdf}  
\caption{Hubble diagrams for a realization drawn from Set 1 (left)
and Set 2 which contains  a second non-Ia population (right).
Type~Ia's are shown in blue and non-Ia's in red, those that fall below detection threshold
are transparent.  The green X's represent the 2\% among those without spectroscopy that
are assigned an incorrect redshift.
   \label{hd:fig}}
\end{figure}

\begin{figure}[htbp] %  figure placement: here, top, bottom, or page
   \centering
   \includegraphics[scale=0.4]{/Users/akim/project/abc/doc/seed2_hist.pdf}
   \includegraphics[scale=0.4]{/Users/akim/project/abc/doc/seed2_pop2_hist.pdf}  
\caption{Histogram of redshifts of discovered SNe~Ia and non-Ia from Set 1 (left)
and Set 2  (right).
   \label{hist:fig}}
\end{figure}

The generative model for the second realized data set (Set 2)
is a perturbed version of that used to make Set 1, and is different from the analysis model.
A second non-Ia population is introduced with average luminosity 0.5 mag fainter
than the Ia's and with 0.25 mag dispersion.
The relative rate of the original non-Ia population relative to the second is quadratic
in redshift:
\begin{equation}
1 -\frac{0.8}{\left(1.1\times z_{max}\right)^2}z^2.
\end{equation}
A realization of Set 2 is shown in Figure~\ref{hd:fig}.
The same seeds used in the construction of Set 1 are used:  The salient differences
between the two realizations are that a fraction of non-Ia's have absolute magnitude
based on the second population's average luminosity and intrinsic dispersion; from
that population different transients pass detection threshold; spectroscopic follow-up is redistributed
for the new set that is detected.

Sets 1 and 2 share identical discovered SNe~Ia.  For reference, this pure SN~Ia
set is analyzed with 100\% spectroscopic typing. 


\subsection{Analysis Results}
The simulated data Sets 1 and 2 are analyzed using the model described
in \S\ref{model:sec}.  The analysis is performed using STAN 
\citep{stan-software:2015},
a probabilistic programming language for
inference of Bayesian models with Hamiltonian Monte Carlo
using the No-U-Turn sampler \citep{Homan:2014:NSA:2627435.2638586}.  Each run of the analysis
uses 24 chains each with 1000 links.  In warmup, the first half of the links are used to
determine parameters for the integration step size and the ``mass''
parameters of the Hamiltonian. The complementary
12,000 links represent draws from the model-parameter posterior.

In lieu of including data of
a pure large SN~Ia sample at low-redshift.
supplemental priors are applied to the SN~Ia absolute magnitude and intrinsic dispersion
\begin{align}
\alpha_{SNIa} & \sim \mathcal{N}_{\ln}\left(\ln\left(\alpha^0_{SNIa}\right),\frac{\ln{10}}{2.5}0.02\right)\\
\sigma_{SNIa} & \sim \mathcal{N}_{\ln}\left(\ln\left(0.1\right),0.1\right),
\end{align}
where $\alpha^0_{SNIa}$ is the true input luminosity.


The output of the analyses are presented as contours of the model
parameter pdf's for the case of 100\% spectroscopic
typing in Figure~\ref{hd:fig}, and as the mean, median, and
credible intervals of $w$ for different fractions of
spectroscopic completeness in Table~\ref{seed2:tab}.
For Set 1 the generative and analysis models
are identical. Consequently the input parameters lie 
within the 68\% credible intervals of the resultant posterior despite the non-trivial appearance of
the Hubble diagram in Figure~\ref{hd:fig}.
The decrease in spectroscopic coverage leads to shifts in $w$ of 0.04 relative to the 100\%-coverage case, which is small
relative to the credible interval range of 0.281.  However, there is an
increase in parameter uncertainties, going from $\Delta w= 0.281$ to 0.309
when there is no spectroscopy.
At the opposite extreme of no spectroscopic classification the $w$ interval inflates to
$\Delta w = 0.309$, a constraining result considering that none of the
objects are directly typed and redshifted.
The Set 1 analysis gives a best-fit $w$ marginally smaller than $-1$ while its
subset of SNe~Ia gives a results marginally larger than $-1$.  The addition of the extra non-Ia's provides no appreciable leverage on the uncertainty of $w$.

\begin{figure}[htbp] %  figure placement: here, top, bottom, or page
   \centering
   \includegraphics[scale=0.17]{/Users/akim/project/abc/results/seed2/{contour.2000.1.0}.pdf}
   \includegraphics[scale=0.17]{/Users/akim/project/abc/results/seed2_pop2/{contour.2000.1.0}.pdf} 
\caption{Probability distribution functions of the model parameters for the case of
 100\% spectroscopy for Model 1 (left) and Model 2 (right).
    Solid lines represent the input parameters.  Dashed lines in the
   histograms are the 0.16 and 0.84 quantiles. 
   \label{contour1:fig}}
\end{figure}

\begin{table}
\centering
\begin{tabular}{|c|cc|ccc|}
\hline
Fraction Spec&$<w>$ & $w$ median &\%-ile  &$w$ range & $\Delta w$\\ \hline
\multicolumn{6}{|c|}{SN Ia only}\\ \hline
$1.00$& $-0.94 \pm 0.09$ &$-0.94$ &$0.68$ & $[-1.076, -0.801]$ & $0.275$ \\
&  &  & $0.90$ & $[-1.166, -0.718]$ & $0.448$ \\
&  &  & $0.95$ & $[-1.208, -0.682]$ & $0.525$ \\
 \hline
\multicolumn{6}{|c|}{Set 1}\\ \hline
$1.00$ &$-1.06 \pm  0.09$ &$-1.05$ &$0.68$ & $[-1.196, -0.914]$ & $0.281$ \\
&&&$0.90$ & $[-1.292, -0.826]$ & $0.466$ \\
&&&$0.95$ & $[-1.340, -0.784]$ & $0.556$ \\
\hline
$0.60$ &$-1.08 \pm  0.10$ & $-1.08$ &$0.68$ & $[-1.227, -0.933]$ & $0.293$ \\
& & &$0.90$ & $[-1.331, -0.836]$ & $0.495$ \\
& & &$0.95$ & $[-1.385, -0.787]$ & $0.598$ \\
\hline
$0.20$ &$-1.10 \pm  0.10$ & $-1.09$ &$0.68$ & $[-1.249, -0.943]$ & $0.305$ \\
& & &$0.90$ & $[-1.354, -0.851]$ & $0.502$ \\
& & &$0.95$ & $[-1.408, -0.810]$ & $0.597$ \\
\hline
$0.00$ &$-1.07 \pm  0.10$ & $-1.06$ &$0.68$ & $[-1.218, -0.909]$ & $0.309$ \\
& & &$0.90$ & $[-1.320, -0.812]$ & $0.508$ \\
& & &$0.95$ & $[-1.372, -0.770]$ & $0.602$ \\
\hline
\multicolumn{6}{|c|}{Set 2}\\ \hline
$1.00$ &$-0.94 \pm  0.09$ &$-0.94$ &$0.68$ & $[-1.078, -0.804]$ & $0.274$ \\
&&&$0.90$ & $[-1.168, -0.720]$ & $0.447$ \\
&&&$0.95$ & $[-1.214, -0.680]$ & $0.534$ \\
\hline
$0.60$ &$-1.04 \pm  0.10$ & $-1.03$ &$0.68$ & $[-1.182, -0.883]$ & $0.298$ \\
& & &$0.90$ & $[-1.285, -0.791]$ & $0.494$ \\
& & &$0.95$ & $[-1.332, -0.749]$ & $0.583$ \\
\hline
$0.20$ &$-1.10 \pm  0.10$ & $-1.09$ &$0.68$ & $[-1.247, -0.936]$ & $0.311$ \\
& & &$0.90$ & $[-1.359, -0.840]$ & $0.518$ \\
& & &$0.95$ & $[-1.413, -0.798]$ & $0.614$ \\
\hline
$0.00$ &$-1.11 \pm  0.10$ & $-1.10$ &$0.68$ & $[-1.266, -0.948]$ & $0.317$ \\
& & &$0.90$ & $[-1.375, -0.850]$ & $0.525$ \\
& & &$0.95$ & $[-1.428, -0.808]$ & $0.621$ \\
\hline
\end{tabular}
\caption{Mean, median, and credible intervals of $w$ for the SNe~Ia only,
Model 1, and Model 2 for varying
fractions of spectroscopic coverage. \label{seed2:tab}}
\end{table}

The generative model for Set 2 differs from the analysis model.
While it is expected for the input and resultant non-Ia parameters not to match, 
when there is 100\% spectroscopic coverage the posteriors of the other
parameters accommodate the input with the $w$ posterior matching
closely that of the SN~Ia-only analysis.  Unlike Model~1, which exhibits no bias
in $w$ with decreasing spectroscopy, Model~2 does have a shift in the $w$
credible region.  
The values of the mean and median values of
$w$ for both 20\% and 0\% coverage are no outside the 68\%
credible region of the ideal 100\% coverage.
The Set 2 $\Delta w$ increases quicker with decreasing spectroscopy
relative to Set 1.

These results are consistent with the trivial expectation that when generative
and analysis models are identical statistically correct parameter determinations
occur, and that analysis with an incorrect non-Ia model can lead to bias.
The latter case is mitigated with complete spectroscopic typing, though then
the non-Ia's contribute little to the $w$ constraints. 

%
%\subsubsection{SN~Ia Only}
%A classical analysis would only include confirmed SNe~Ia with a definitive
%redshift. The realization of simulated data has 415 Type~Ia supernovae.
%Results of the analysis of a pure SN~Ia sample with no
%host misclassification are presented:  Figure~\ref{contour.ia_only:fig}
%shows the posteriors for  $\Omega_M$--$w$ and Table~\ref{ia_only:tab} the
%credible equal-tailed intervals for $w$.  The true $w=-1$ is within
%the 0.58 credible interval. (Note that $w$ is the parameter
%of most interest for the experiment.)
%
%\begin{figure}[htbp] %  figure placement: here, top, bottom, or page
%   \centering
%   \includegraphics[scale=0.55]{/Users/akim/project/abc/results/{contour..ia_only.500}.pdf} \newline
%\caption{Probability distribution function for $\Omega_M$--$w$ for only SNe~Ia
%   all with spectra.
%   Solid lines represent the input cosmology.  Dashed lines in the
%   histograms are the 0.16 and 0.86 quantiles. 
%   \label{contour.ia_only:fig}}
%\end{figure}
%
%\begin{table}
%\centering
%\begin{tabular}{|c|cc|}
%\hline
%\%-ile & $w$ range & $\Delta w$\\ \hline
%$0.68$ & $[-1.254, -0.975]$ & $0.279$ \\
%$0.90$ & $[-1.347, -0.889]$ & $0.458$ \\
%$0.95$ & $[-1.396, -0.846]$ & $0.550$ \\
%\hline
%\end{tabular}
%\caption{Credible intervals of $w$ using the SNe~Ia only with spectroscopic
%observations.\label{ia_only:tab}}
%\end{table}
%
%\subsubsection{Full Sample: Varying Spectroscopic Completeness}
%We now turn to the full sample of SNe~Ia and non-Ia's.  Recall that our simple
%model treats the non-Ia's in the sample
%as standard candles with a large (1 mag) intrinsic dispersion.
%
%Results for the $\Omega_M$--$w$ posterior for the
%case of 40\%, 70\%, and 100\% spectroscopic completeness
%are shown in Figures~\ref{contour.200:fig}, \ref{contour.350:fig}, and
%\ref{contour.500:fig} respectively.
%The credible intervals
%of $w$ for are given in Table~\ref{compare:tab}, with
%the input $w=-1$ within the 0.65, 0.49, and 0.47  intervals
%respectively.
%
%\begin{figure}[htbp] %  figure placement: here, top, bottom, or page
%   \centering
%   \includegraphics[scale=0.40]{/Users/akim/project/abc/results/{contour.200}.pdf} 
%   \caption{Probability distribution function for $\Omega_M$--$w$ for:
%    SNe~Ia and non-Ia with 40\%  spectroscopic completeness.
%   \label{contour.200:fig}}
%\end{figure}
%
%\begin{figure}[htbp] %  figure placement: here, top, bottom, or page
%   \centering
%   \includegraphics[scale=0.40]{/Users/akim/project/abc/results/{contour.350}.pdf} 
%   \caption{Probability distribution function for $\Omega_M$--$w$ for
%    SNe~Ia and non-Ia with 70\%  spectroscopic completeness.
%   \label{contour.350:fig}}
%\end{figure}
%
%\begin{figure}[htbp] %  figure placement: here, top, bottom, or page
%   \centering
%   \includegraphics[scale=0.40]{/Users/akim/project/abc/results/{contour.500}.pdf} 
%   %\includegraphics[scale=0.55]{/Users/akim/project/abc/results/{contour.200}.pdf}    
%   \caption{Probability distribution function for $\Omega_M$--$w$ for
%    SNe~Ia and non-Ia with 100\%  spectroscopic completeness.
%   \label{contour.500:fig}}
%\end{figure}
%
%\begin{table}[htbp] 
%\centering
%\begin{tabular}{|cc|cc|}
%\hline
%\# Spectra & \%-ile & $w$ range & $\Delta$\\ \hline
%200& $0.68$ & $[-1.284, -0.991]$ & $0.293$ \\
%200& $0.90$ & $[-1.385, -0.900]$ & $0.485$ \\
%200& $0.95$ & $[-1.435, -0.860]$ & $0.575$ \\
%350& $0.68$ & $[-1.243, -0.952]$ & $0.291$ \\
%350& $0.90$ & $[-1.339, -0.858]$ & $0.481$ \\
%350& $0.95$ & $[-1.390, -0.817]$ & $0.573$ \\
%500& $0.68$ & $[-1.236, -0.948]$ & $0.289$ \\
%500& $0.90$ & $[-1.336, -0.861]$ & $0.475$ \\
%500& $0.95$ & $[-1.385, -0.815]$ & $0.570$ \\
%\hline
%\end{tabular}
%\caption{Credible intervals of $w$ given different levels of spectroscopy.  \label{compare:tab}}
%\end{table}
%
%As would be expected, the better the spectroscopic completeness, the less bias
%and smaller uncertainties that are obtained.
%
%
%\section{Subsections of the Model}
%
%\subsection{Type}
%\label{type:sec}
%Since our transient sample may not consist purely of SNe~Ia, the model
%has a term of the form $P(T, G | z,\mu)$, which should account for all types of
%objects that could potentially be mistaken for Type~Ia:
%the rates and association with host galaxies are described for
%the full population of potential interlopers.
%This term is not currently well known, particularly at the highest redshifts probed by DES.
%
%An idea is to consider two types: SN~Ia and non-SN~Ia, where the latter's
%intrinsic distributions have loose priors.  Transients that have been spectroscopically
%classified as non-Ia will provide the strongest leverage in constraining the
%contamination: this subset must be unbiased or properly weighted
%relative to the underlying population that constitutes the Hubble diagram.
%Running the model on a pure sample of SNe~Ia can have benefits.
%As discussed in \S\ref{systematics:sec}, comparing these results with
%those from the full sample provides
%a test of systematics.  Alternatively, the non-Ia model may be better constrained
%with the pure sample.
%
%
%To minimize the contribution of non-Ia contamination we strive for
%a pure sample selection so that $P(T=\text{non-Ia}| \mu) \rightarrow 0$.
%
%
%\subsection{Host Matching}
%\sloppy
%The host-galaxy properties depend on the GC and the transient
%coordinates $P({z}_H, {\theta}_H | {\mathit{Gals.}}, \text{RA/Dec})$.  With a
%spectrum of the transient and the underlying background, identification of the host galaxy
%is usually straightforward.
%Otherwise, projections or ambiguous cases can result in the misidentification of
%the host.
%Other supernova surveys (e.g.\ SNLS) 
%with spectroscopically confirmed host associations (e.g.\ with
%matched transient and galaxy redshifts) that replicate the DES population
%can be used to put a prior on this probability.
%
%%An effective way of addressing this term is to
%%analyze the supernovae assuming a correct host match, and
%%experimentally determine $P(\text{mismatch} )$ and
%%$\left. \langle {\mu}-\mu(z)\rangle \right|_{\text{mismatch}}$, $\left. \langle {z}-z \rangle \right|_{\text{mismatch}}$ and cross-terms to correct bias in the Hubble diagram. 
%
%
%% only model $T=\text{SN~Ia}$,
%%and experimentally determine $P(\text{non-Ia})$, 
%%$\left. \langle \mu-\mu_{model} \rangle \right|_{non-Ia}$, and its uncertainty for the
%%sample of non-Ia's that are discovered and misclassified as SN~Ia.
%%This measurement is achieved through spectroscopic
%%typing of a subset drawn from the populations used in the Hubble diagram.
%%The weighted bias would be added as a correction to the Hubble diagram under the assumption that all objects are SNe~Ia.
%
%
%\section{Systematic Tests}
%\label{systematics:sec}
%The Hubble diagrams of subsets of data quantify systematic uncertainty.  
%Subsets that may be expected to show evidence for systematics include:
%spectroscopically typed SNe~Ia; transients in deep (3) versus shallow (1/2) DES fields;
%transients with NIR data; splits based on host properties.
%
%A potentially more probative test is to compare the distance moduii inferred for the same
%supernovae, but reducing the amount of data considered.  A subset of DES
%transients have higher signal-to-noise, spectroscopic confirmation, and NIR coverage than
%the average transient, we can take away that extra information in calculating distance modulus
%to get $P(\mu | \mathit{data} - \mu | \mathit{data}^-, z | \mathit{data} - z | \mathit{data}^-)$.
%It is possible that differences in $\mu$s and $z$s have smaller uncertainties that the
%$\mu$ and $z$ alone due to common measurement uncertainties. 
%
%
%
%Generically ${{T}}$ can describe sample selection.  I choose to focus
%on objects typed as SN~Ia, rather than  all discovered transients, not just those typed as SN~Ia. 
%The non-Ia transient population has significant model uncertainty and the information they impart on the expansion history is weak:  I anticipate it is cleaner to limit ourselves to
%modeling discoveries we think are SN~Ia. This restriction does not obviate the need
%to consider non-Ia contamination in the sample, rather
%we only need to model non-Ia's
%that are mistaken for Ia and not all discovered transients. 



\section{Model}
We infer a Hubble diagram based on measurements of discovered transients
that have passed some selection criteria (e.g.\ classified as Type~Ia supernovae).  The posterior
of the Hubble diagram is denoted as
\begin{equation}
p({\mu},{z} |  {{ADU}}, {{T}}_S,{{z}}_S,
{{\theta}}_S, \theta_G, \tau).
\label{hd:eqn}
\end{equation}
The set of distance moduli
and redshifts $(\mu, z)$ contains the points that constitute the Hubble diagram.  The
data
from which the Hubble diagram is inferred include:
the transient photometry ${ADU}$; spectral measurements of
transient
type, redshift and SN-parameters ${T}_S$, ${z}_S$, ${\theta}_S$;
host-galaxy parameter measurements $\theta_H$ (including redshift,
mass, sSFR, etc.).  
The subset transients that enter the cosmology
analysis is based on the sample-selection criteria $\tau$: for example
the classification score from PSNID (a classifier based on multi-band
light curves).

The posterior of $\mu$ and $z$ is an efficient way
to distribute supernova results in is a way that is independent of cosmological model,
facilitating joint-probe analysis or consideration of non-standard models.  Choosing
a specific cosmological model, the posterior of the model parameters $\theta_\mu$
is of interest.  The model presented here accommodates both options.

A sketch of a more extensive model that includes the
model of Equation~\ref{hd:eqn} is depicted in the Probabilistic Graphical Model
(PGM)
shown in Figure~\ref{pgm:fig}.
The fundamental parameters are those with no incoming arrows;  most are
used to model the universe: 
the background cosmology, transients that could enter the sample, and their host galaxies.
\begin{itemize}
\item  $g$ labels the  galaxy that hosts the transient.
\item $\theta_\mu$ are the parameters that specify that distance modulus.  They could
be the parameters of a specific cosmological
parameters, or an empirical model independent of physics.
\item $\theta_\tau$ govern the relative rates of supernovae of different types in galaxies.
\item $\theta_T^{Ia}$, $\theta_{Ti}^{Ia}$.  The model parameters for SNe~Ia, including
parameters that describe distributions.
Different SN~Ia models could be considered.  The treatment of SALT2
has been presented in \citet{2011MNRAS.418.2308M}:
in this model each supernova
has individual $c$ and $x_1$ parameters that constitute $\theta_{Ti}^{Ia}$;
global SN~Ia properties
$\alpha$, $\beta$, $\sigma_{int}$ 
 and  underlying  $c$ and $x_1$ distributions   constitute $\theta_T^{Ia}$.
\item $\theta_T^{\mathit{non-Ia}}$, $\theta_{Ti}^{\mathit{non-Ia}}$.  False-positive non-Ia transients
that enter the analysis are described by a parameterized model.
\item The transmission functions $\phi(\lambda)$ describe the optical path from the
atmosphere to counts.  Calibration solutions are generally provided
but are included in Figure~\ref{pgm:fig} to encompass the case of self-calibration.
\end{itemize}

\begin{figure}[htbp] %  figure placement: here, top, bottom, or page
   \centering
   \includegraphics[width=7in]{/Users/akim/project/abc/results/hdpgm.pdf} 
   \caption{Probabilistic Graphical Model for the SN~Ia analysis.  
   The forward modeling
   flow goes from left to right, starting with the description of the universe, observatory,
   and data.    A transient has host galaxy $g_i$, which determines its redshift $z_i$
   and galaxy parameters $\theta_{gi}$.
   A model with parameters $\theta_\mu$ fix the distance modulus $\mu_i$.
   The transient type $T$ depends on the host-galaxy parameters  $\theta_{gi}$
   and rates $\theta_r$.   The transient
   parameters $\theta_T^X$, $\theta_{Ti}^X$ (and perhaps the host galaxy) determine the luminosity $L$.       The 
   incoming photon flux $n$, $n_g$  are then fixed
   with redshift and distance modulus.
   The instrumental transmission function $\phi$ is calibrated with data ${Z}$ and
   gives the expected
   counts $\overline{f}$, $\overline{f}_g$. 
   The transient has realized light curves (${ADU}$) and passes selection criteria
   through $\tau_i$ to be included in the sample.  Some transients have spectral data
   ${X}_S$.  The properties of the galaxy identified as the host galaxy is $\theta_G$. 
   \label{pgm:fig}}
\end{figure}

From these fundamental parameters we can determine the expected
signals
through a series of intermediary parameters,
some of which are deterministic (points in Figure~\ref{pgm:fig}) and others
probabilistic (ovals).
\begin{itemize}
\item $z_i$, $\theta_{gi}$ are the galaxy redshift and parameters (e.g.\ mass, sSFR, metallicity).
Each galaxy $g_i$ has its values for these quantities.
\item $\mu_i=\mu(z_i; \theta_\mu)$.  The distance modulus is a function of redshift,
Unspecified at the moment,  it could
be a $\Lambda$CDM prediction with cosmological parameters, a spline with knot values
as parameters.
The redshift of the galaxy is used in the Hubble diagram, peculiar velocities are not considered.
\item $P(T_i | \theta_r, \theta_{gi})$.  The transient type $T_i$, SN~Ia or non-Ia, depends
on the relative rates given the host properties.
\item $P(L_i(t,\lambda)| T_i, \theta_T^{Ia}, \theta_{Ti}^{Ia}, \theta_T^{\mathit{non-Ia}}, \theta_{Ti}^{\mathit{non-Ia}},
\theta_{gi})$.  The source model determines
the luminosity; the model may include intrinsic dispersion to make the luminosity
probabilistic. The  model includes  information on the
source and line-of-sight effects, which specify the effective SED.   Only the
parameters of the model of the appropriate type $T$ are relevant.  The galaxy parameters
$\theta_{gi}$ are included if correlations with host-galaxy properties are to be modeled.
\item $n_i(t,\lambda; L_i, \mu_i, z_i)$, $n_{gi}(t; g_i, \mu_i, z_i)$.  The  fluxes of
the transient and host that are incident at Earth
are fixed by the luminosity, distance modulus, and redshift.
The supernova redshift is required for this calculation,
for simplicity the supernova and host-galaxy redshift are assumed to be equal.
\item $\overline{f}$ and
$\overline{f}_g$ are the expected counts of the transient and galaxy respectively,
coming from $\int n_i \phi d\lambda$.  
\end{itemize}

The observations are as follows:
\begin{itemize}
\item ${f}$ is the realized flux.  It depends on the underlying flux and
the photometric noise.
\item $\theta_G$ represents the galaxy parameters of the
galaxy catalog used to identify prospective host galaxies.
\item ${T}_S$, ${z}_S$, ${\theta}_S$ are the transient type, redshift and properties from
spectroscopic data. Details of the actual measurement are ignored and a direct association
is made with the fluxes of the transient and underlying galaxy $n$, $n_g$.
\item $\tau$ is the score used to decide if the transient is used in the analysis.  It is
determined based on other measurements $\tau(f,T_S)$.
\item ${Z}$ represents the calibration measurements.   Although this may be given
by the Calibration Task Force, the PGM is written for the more general case where the calibration
is modeled within our analysis.
\end{itemize}

%The likelihood for the model is
%\begin{multline}
%p(\mathit{ADU}, {{T}}_S,{{z}}_S,
%{{\theta}}_S, \theta_G, \tau, Z | g, \theta_\mu, \theta_r, \theta_T^{Ia}, \theta_{Ti}^{Ia}, \theta_T^{non-Ia}, \theta_{Ti}^{non-Ia}, \phi) = \\
%p(\theta_G|g)p(Z|\phi)p(\mathit{ADU}, {{T}}_S,{{z}}_S,
%{{\theta}}_S, \tau | g, \theta_\mu, \theta_r, \theta_T^{Ia}, \theta_{Ti}^{Ia}, \theta_T^{non-Ia}, \theta_{Ti}^{non-Ia}, \phi).
%\label{likelihood:eqn}
%\end{multline}

The simulation of data from the model presented in
and their analysis is implemented in Python and are available
on GitHub\footnote{{git@github.com:AlexGKim/abc.git}}.

\bibliographystyle{elsarticle-num} 
\bibliography{/Users/akim/Documents/alex}

\end{document}

